\documentclass[12pt, a4paper]{article}
\usepackage[a4paper, includeheadfoot, mag=1000, left=2cm, right=1.5cm, top=1.5cm, bottom=1.5cm, headsep=0.8cm, footskip=0.8cm]{geometry}
% Fonts
\usepackage{fontspec, unicode-math}
\setmainfont[Ligatures=TeX]{CMU Serif}
\setmonofont{CMU Typewriter Text}
\usepackage[english, russian]{babel}
% Indent first paragraph
\usepackage{indentfirst}
\setlength{\parskip}{5pt}
% Diagrams
\usepackage{graphicx}
\usepackage{float}
% Page headings
\usepackage{fancyhdr}
\pagestyle{fancy}
\renewcommand{\headrulewidth}{0pt}
\setlength{\headheight}{16pt}
%\newfontfamily\namefont[Scale=1.2]{Gloria Hallelujah}
\fancyhead{}

\usepackage{amsmath}

\graphicspath{ {images/} }

\usepackage{listings}
\lstdefinestyle{lablisting}{
  basicstyle=\scriptsize\ttfamily,
  numbers=left,
  stepnumber=1,
  otherkeywords={EOF, O\_RDONLY, STDIN\_FILENO, STDOUT\_FILENO, STDERR\_FILENO},
  numbersep=10pt,
  showspaces=false,
  showstringspaces=false
}

\newcommand{\specialcell}[2][l]{%
  \begin{tabular}[#1]{@{}l@{}}#2\end{tabular}}

\begin{document}

% Title page
\begin{titlepage}
\begin{center}

\textsc{Национальный исследовательский университет ИТМО}
\vfill
\textbf{Лабораторная работа №1\\[4mm]
по дисципение Метрология, стандартизация и сертификация\\[16mm]
}
%\textit{Вариант 11\\[16mm]}
\begin{flushright}
Студент: Саржевский Иван
\\[2mm]Группа: P3302
\end{flushright}
\vfill
г. Санкт-Петербург\\[2mm]
2021 г.

\end{center}
\end{titlepage}

\section*{Задание}

Записать оценку измеряемой величины с учетом случайной и
систематической погрешностей, если производились прямые измерения.

\section*{Измерения}

\begin{center}
\begin{tabular}{|c|c|}
  \hline
  N & Значение, мм \\\hline
  1 & 1.2 \\\hline
  2 & 1.3 \\\hline
  3 & 1.4 \\\hline
  4 & 1.3 \\\hline
  5 & 1.3 \\\hline
\end{tabular}
\end{center}

\section*{Ход работы}

\subsection*{Устранение или учет известных систематических погрешностей}

О системных погрешностях ничего не известно, поэтому переходим к пункту 2.

\subsection*{Вычисление среднего значения}

За эту оценку принимают среднее арифметическое значение  по формуле:

$$\bar{x} = \frac{1}{n} * \sum_{i = 1}^{n} x_i$$

$\bar{x} = 1/5 * (1.2 + 1.3 + 1.4 + 1.3 + 1.3) = 1.3$ мм.

\subsection*{Вычисление среднего квадратического отклонения}

$$S_x = \sqrt{\frac{1}{n-1} * \sum_{i = 1}^{n} (x_i - \bar{x})^2}$$

$S_x = \sqrt{\frac{1}{4} * \sum_{i = 1}^{n} (x_i - 1.3)^2} = 0.0707$ мм.

\subsection*{Среднеквадратическое отклонение среднего арифметического}

$$S_{\bar{x}} = \frac{S}{\sqrt{n}}$$

$S_{\bar{x}} = 0.0707 / \sqrt{5} = 0.0316$ мм.

\subsection*{Исключение грубых погрешностей}

$$G_1 = \frac{|x_{max} - \bar{x}|}{S}; G_2 = \frac{|\bar{x} - x_{min}|}{S}$$

$G_1 = \frac{|1.4 - 1.3|}{0.0707} = 1.414$

$G_2 = \frac{|1.2 - 1.3|}{0.0707} = 1.414$

$G_T = 1.715$ для $q = 5\%$ и пяти измерений.

$G_1 \leq G_T$, поэтому $x_{max}$ не считаем промахом.

$G_2 \leq G_T$, поэтому $x_{min}$ не считаем промахом.

\subsection*{Доверительные границы случайной погрешности}

$$\epsilon = t S_x, t [P = 95\%; n = 5] = 2.776$$

$\epsilon = 0.0316 * 2.776 = 0.0877$ мм.

\subsection*{Учет систематической погрешности}

$\theta = 0.1$ мм, согласно надписи на приборе.

\subsection*{Учет полной абсолютной погрешности прямого измерения}

\subsubsection*{Абсолютная погрешность}

$$\Delta \bar{x} = \sqrt{\epsilon^2 + \theta^2}$$

$\Delta \bar{x} = \sqrt{0.0877^2 + 0.1^2} = 0.133$ мм.

\subsubsection*{Относительная погрешность}

$$\delta x = \frac{\Delta \bar{x}}{\bar{x}} * 100\%$$

$\delta x = \frac{0.133}{1.3} * 100\% = 10.23\%$

\subsection*{Запись результата}

$$x \approx 1.30 \pm 0.13 \text{мм.}$$

\section*{Вывод}

В результате выполнения данной лабораторной работы была произведена
запись оценки измеряемой величины с учетом случайной и систематической
погрешностей по результатам прямых измерений.

\end{document}
